\documentclass[fleqn, a4paper, russian]{extarticle}

\usepackage[utf8]{inputenc}
\usepackage[T1, T2A]{fontenc}
\usepackage[english, main = russian]{babel}
\usepackage[style=russian]{csquotes}
\usepackage{pscyr}
\usepackage{indentfirst}
\parindent 0.0cm
\usepackage{graphicx}
\usepackage{natbib}
\usepackage{caption, subcaption}
\usepackage[top=2cm, left=2cm, right=2cm, left=2cm]{geometry}
\usepackage{amsmath}
\usepackage{amssymb}
\usepackage{ragged2e}
\usepackage{adjustbox}
\usepackage{makecell}
\usepackage{multirow}
\usepackage{wrapfig}
\usepackage{threeparttable}
%\usepackage{enumerate}
\usepackage{enumitem}
\usepackage{float}
%\usepackage{epstopdf}
%	\epstopdfsetup{suffix=}
\usepackage{multicol}
\usepackage[hidelinks]{hyperref}

%\usepackage{listings}
%\usepackage{inconsolata}

%\usepackage{pgfplots}
%	\pgfplotsset{compat=1.14}

%\lstset{
%	basicstyle = \ttfamily,
%}

\usepackage{textcomp}
\usepackage{tikz}
	\usetikzlibrary{shapes, arrows, arrows.meta}

\graphicspath{{images/}}

\captionsetup[figure]{name = Рисунок, labelsep = endash}
\captionsetup[table]{name = Таблица, labelsep = endash, justification=raggedright, singlelinecheck=false}
%\setlength{\mathindent}{0pt}

\newcommand{\R}{\mathbb{R}}

\makeatletter
\setlength\@fptop{0\p@}
\makeatother

\newcolumntype{L}[1]{>{\raggedright\let\newline\\\arraybackslash\hspace{0pt}}m{#1}}
\newcolumntype{C}[1]{>{\centering\let\newline\\\arraybackslash\hspace{0pt}}m{#1}}
\newcolumntype{R}[1]{>{\raggedleft\let\newline\\\arraybackslash\hspace{0pt}}m{#1}}

\newenvironment{resumeBlock}[1]
	{
		\vspace{1em}
		{\bfseries #1}
		\vspace{0.25em}
		\hrule
		\vspace{0.5em}
	}
	
	{
		%
	}

\begin{document}
	\pagestyle{empty}
	\fontsize{14pt}{0.4em}\selectfont
	{ % Header
		\begin{minipage}[l]{0.35\linewidth}
			\RaggedRight
			%\hspace{1cm}
			{\fontsize{11pt}{0em}\selectfont \url{www.github.com/Niirn-dev} }
		\end{minipage}
		\hfill
		\begin{minipage}[l]{0.25\linewidth}
			\Centering
			{\fontsize{16pt}{0.4em}\selectfont 
			 \bfseries Алякин Сергей}
		\end{minipage}
		\hfill
		\begin{minipage}[l]{0.3\linewidth}
			\RaggedLeft
			Санкт-Петербург\\
			+7-(911)-264-44-83\\
			dreammagica@gmail.com
		\end{minipage}
	}

	\begin{resumeBlock}{Образование}
		{\bfseries Университет ИТМО}\\%\vspace{-1em}
		\begin{enumerate}[label=--,topsep=0pt,itemsep=0pt]
			\item {\bfseries Степень магистра:} управление в технических системах.
			\item {\bfseries Полученные навыки и знания:} программирование систем реального времени, линейная алгебра и мат. анализ, теория автоматического управления, первичная обработка данных, устройство реляционных СУБД.
			\item {\bfseries Иностранные языки:} Английский на уровне Advanced (CEFR C1)
		\end{enumerate}
	\end{resumeBlock}
	
	\begin{resumeBlock}{Опыт работы}
		{\bfseries CS IT, программист \hfill Июль-Август 2018}\\
		Разработка программы для автоматического формирования отчёта по информации из SQL базы данных.
		\begin{enumerate}[label=--,topsep=0pt,itemsep=0pt]
			\item Финальное решение представлено в виде надстройки для Microsoft Excel.
			\item Выполнено на C\# при помощи VSTO и .Net 3.5.
			\item Возможность подключения к БД в сети компании.
		\end{enumerate}
		\vspace{0.3cm}
		{\bfseries Германия, KIT. Стажёр \hfill Ноябрь 2018}\\
		\begin{enumerate}[label=--,topsep=0pt,itemsep=0pt]
			\item Реализовал процесс получения информации о положении при помощи спутников на реальной системе.
			\item Реализовал алгоритмы распознания объектов на изображении.
			\item Получил опыт работы с иностранными коллегами на английском языке.
		\end{enumerate}
	\end{resumeBlock}

	\begin{resumeBlock}{Проекты}
		{\bfseries Виртуальная система управления реального времени}\\%\vspace{-1em}
		\begin{enumerate}[label=--,topsep=0pt,itemsep=0pt]
			\item Программа написана в Qt на C++.
			\item Реализованы непрерывная и дискретная системы.
			\item Вывод графиков переходных процессов в реальном времени.
		\end{enumerate}
	
		{\bfseries Программа для обработки звука}\\
		\begin{enumerate}[label=--,topsep=0pt,itemsep=0pt]
			\item Программа написана в Qt на C++.
			\item Реализована возможность применения фильтров с любой физически реализуемой передаточной характеристикой.
		\end{enumerate}
	\end{resumeBlock}

	\begin{resumeBlock}{О себе}
		Возраст: 24.\\
		Личные качества: пунктуальный, исполнительный, вежливый, аккуратный.
	\end{resumeBlock}
	
	\begin{resumeBlock}{Навыки}
		C++, C\#, \LaTeX, Matlab, SQL, Git
	\end{resumeBlock}
	
	\if 0
	\begin{resumeBlock}{Education}
		Qwer
	\end{resumeBlock}

	\begin{resumeBlock}{Employment}
		Qwer
	\end{resumeBlock}

	\begin{resumeBlock}{Software Projects}
		Qwer
	\end{resumeBlock}

	\begin{resumeBlock}{Skills}
		Qwer
	\end{resumeBlock}
	\fi
\end{document}